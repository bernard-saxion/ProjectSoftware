\chapter{Project organisation}

\section{Team members}

The team consists of four members, as listed in table \ref{tbl:contactdetails}. All members are first-year students within Saxion, so it is a small team with little experince, as such team members will not bear official titles or have authority over certain parts of the project. Instead, the tasks will be allotted based on availibility, confidence, and the results of previous tasks. Every member bears responsibility over every part of the project. While normally this would be unacceptable, the small size of the project allows every member to excert tight social control over all the other members, thus if one member fails, the others failed to check up on that member.

\begin{table}
\centering
\caption{Team contact details}
\begin{tabular}{lc r @{@} l}
\\
Name & Student ID & \multicolumn{2}{c}{E-mail address} \\
\hline
Mohammed Musalhi & 402566 & Kx-007&hotmail.com \\
Berend Al & 405759 & 405759&student.saxion.nl \\
Affan Al Hakam & 359751 & affanalhakam&gmail.com \\
Marwan Al-Zadjali & 350231 & m.h.82.8&hotmail.com \\
\end{tabular}
\label{tbl:contactdetails}
\end{table}

There are still two assigned positions, the chair and the scribe, but these are only relevant during the meetings and are not bound to a certain team member: Any member may propose to chair the next meeting and the chair is responsible for assigning a scribe. The chair will prepare and conduct the meeting, whilst the scribe will note down any results and decisions made during that meeting.

\section{Supervision}

The project is supervised four teachers, Christiaen Slot, Ali Yuksel, Erik Karstens, and Andre Fiselier, who also embody the client. They are generally not called to join meetings, but they will be available for assistance during the supervised work hours listed in table \ref{tbl:workhours} and they will review each deliverable for acceptance.

\section{How meetings are organized}
At least twelve hours before each meeting the chair distributes an agenda that lets the members of the group to know what they have to bring, read, or do for the meeting and what will be discussed. Every meeting starts with reviewing the minutes of the last meeting and reviewing the agenda for the current meeting, after that each issue on the agenda will be discussed. Finally, at the end of every meeting we organize the time, date and the goals of next meeting.  

\section{Office hours}

There project is estimated to take about 84 hours, of which some are scheduled so that a proper working environment is guaranteed, some are supervised for when there is a need of guidance or assistance. The hours when we are to work on the project are listed in table \ref{tbl:workhours}. Unscheduled hours are done in squatted workspaces in Saxion Enschede or from home.

\begin{table}
\centering
\caption{Working hours}
\begin{tabular}{lcl}
\\
When & \#Hours & Activity \\
\hline
Weekly & 2 & Scheduled, supervised work \\
Weekly & 2 & Scheduled, unsupervised work \\
Weekly & 4 & Unscheduled, unsupervised work \\
Vacation & ? & Overtime \\
\end{tabular}
\label{tbl:workhours}
\end{table} 

\section{Communication}

Almost all communication will be done by E-mail, both with the members as with the supervisors. Documents are submitted for review by the client through blackboard.

Documents are stored on GitHub\cite{github}, where also the status of project milestones and deliverable are monitored. Often work-in-progress files reside on laptop drives until they are found to be publishable.  The hardware will be stored in one of the lockers of the LDR building.

Document files are distributed in either plain text format (\verb+.txt+), \LaTeX{} source files (\verb+.tex+), or as portable document format (\verb+.pdf+). Software files will be distributed as \verb+C+ or \verb-C++- source format (\verb+.c+, \verb+.cpp+, \verb+.h+). Hardware schematics will be distributed as either NI Multisim's propertary format or in portable document format. Software schematics will be distributed as both UMLet source files (\verb+.uxf+) and one of vector graphics (\verb+.svg+), portable network graphic (\verb+.png+), or portable document (\verb+.pdf+). Archives of multiple compressed files will be distributed as Zip files (\verb+.zip+).

