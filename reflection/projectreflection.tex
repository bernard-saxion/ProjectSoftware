\chapter{Project Reflection}
This chapter will reflect on the project in general, the subject, organisation, planning, what went well and what we can do better in the next project.

\section{Affan Al Hakam}
This project is helping us to know further about how to arduino work with several equipments, and how to organized the group.
 This project went well because our group did this project nicely and every member of the group was active and finished the project on time.
 For the next project, I look forward to make other challenging project and put more creative things in the project.

My opinion about our teamwork in general is very good, and every member of the group is working nicely, and the communication all of us is nice as well.

\section{Berend Al}
I quite enjoyed this project. It was small enough to be managable even if this is our first project, but was still challenging.
I did find the information about the V-Model a bit lacking.
We have only had a couple of lessons about the V-Model and so a lot of details were missing about a lot of phases.
The timeframe may also have been a bit too short to make full use of the V-Model.

We made a good planning and we were mostly able to stick to it, so I'd like to see if we can repeat that trick next project.
Our system designs, on the other hand, were spotty at best which caused us to ignore them in favour of rediscussing implementations during work and just working off of our intuition.

One thing we could improve on the planning is that we fragmented our work too much in the early weeks of the project, we had all the chapters of the project plan and all the subsystem designs done by different people.
Fragmenting the work like this meant that at first the chapters lacked intercoherency and common style.

We seemed to work well as a team, everyone had a sense of duty and would remind and assist others on theirs. Our organisation was rather ad-hoc but that just seemed to reduce friction in switching tasks and helping each other.

\section{Marwan Al-Zadjali}
The main thing I have learned from this project is how to be a responsible person.
Also, I get an idea of how to work in a team which give me an image of how my future work-life would be.
In fact, what I like "the most" in this project is that we are grading ourselves by showing up what we can do, what we cannot do and what we could do if we, for example, have more time.
Personally, I was proud of myself because in these project I used my own life experience to help me doing my works and also came up with good ideas which was helpful to the group.

During the project, the team was combined together.
I think everyone was doing what he was ask to do by the leader of the group.
The team was enjoying the challenges of the project.
The beast thing we did was that we made a projects plan and we follow it just as we planned it.
The project was serious as same as fun, so I hope every team members were enjoying it too.

Also, I would like to thank all the projects advisors and some other teachers for the helpful advises and guidance, I am really appreciate what they did for us.
Finally, I hope next time if the project is a new idea which is not already exist or maybe exist but we have to improve it.

\section{Mohammed Musalhi}
Working in this project was a great experience, as I learnt many functional skills, that would help me through my studies.
The well organized group, and the cooperation between the members led to done the work perfectly and on time.
However, while making the housing there were some problems with compiling the pieces of wood together, because they were not identical.
Nevertheless, that could be improve next time by using the laser cutter.

Teamwork was the best part, where every member share his knowledge with others to improve their skills.

